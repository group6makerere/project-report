
\documentclass{article}
\usepackage{amsmath}
\usepackage{graphicx}
\graphicspath{ {images/} }
 
\begin{document}

	\begin{center}
		{\huge\textbf{MAKERERE UNIVERSITY}}\\
		{\large\textbf{COLLEGE OF COMPUTING AND INFORMATION SCIENCES}}\\
		\textbf{SCHOOL OF COMPUTING AND INFORMATICS TECHNOLOGY}\\
		\textbf{DEPARTMENT OF COMPUTER SCIENCE}
	\end{center}
	GROUP 6 COURSE WORK\\
	COURSE UNIT : RESEARCH METHODOLOGY\\
	INSTRUCTOR : EARNEST MWEBAZE\\
	COURSE CODE : BIT 2207\\



\begin{tabular}{|p{5cm} |p{3cm}|p{3cm}|p{3cm}|}
\hline
\multicolumn{4}{|c|}{Group 6}\\
\hline
\hline
Name & Registration Number & Student Number&Signature\\
\hline
MATOVU GEOFREY&16/U/6902/EVE&216008997&\\
BAMANYE&16/U/4206/EVE&216013268&\\
KAJJOBA MUDASILU&16/U/5198/EVE&216015944&\\
SADALA BADRU ATULINDA&16/U/3901/EVE&216010207\\
\hline

\end{tabular}
\section{HARD DISK FAILURE DETECTOR}
\subsection{Introduction}
The aim of this document is to provide a high light of how hard drive failure prediction can be improved and summarize a range of literature to highlight themes and generate recommendations for future work in this area by Hard Drive Failure Predictor. This section contains the background which purposely describes the origin of hard disk failure and points out past and recent studies, problem statement pointing out research gaps, objectives and scope of the study. It will help clarify to the reader how fulfillment of the research aims and objectives will improve on data storage through effective hard drive failure detection. In addition the document will partitioned into other sections which include the literature review, research methodology timeframe, risks and project budget inclusive.  
\section{PROBLEM STATEMENT}
Data loss especially in developing countries has been a major problem affecting both novice or occasional users and expert users. This is due to the fact that second hand equipment are being imported 95% in these countries. This has led to abrupt disk crashes where Banking institutions, web hosting companies, data centers and personal users have lost vast data beyond recovery. 
Hard disk failure predictor will be a phone based software which will be  a solution to this problem where  all users will use their mobile phones to compute the health status and  durability of  their  logic drives in hardware shops, web hosting companies, banking institutions, homes and any other data storage related institutions. 
\section{OBJECTIVES}
\subsection{GENERAL OBJECTIVE}
To create a system that will effectively predict hard drive defects or failure.
\subsection{SPECIFIC OBJECTIVES}
To critically asses how users maintain and use their hard drives.
To recommend users to detect and predict hard disk failure and maximize data backup.
To evaluate and classify operations of data recovery schemes by different users.

\section{RESEARCH METHODOLOGY}
\subsection{Research Questions} 
Our overall design of our project will take a quantitative approach, based on quantities obtained using a quantifiable measurement process and qualitative approach where data is utilized to test the following research questions: 
Is the system able to detect and predict hard disk failure? 
How do the users maintain and use their hard drives? 
Are users able to back up their information in the event of a hard drive failure? 

\subsection{Data Collection Procedures:} 
Data will be collected by use of observation of conditions under which users utilize and maintain their computers on a daily basis. This is flexible since data is collected where and when an event or activity is occurring. 
Questionnaires will also be formulated to gather data from a wide group of people spread over a wide geographic area. 
In addition Samples will be conducted by selecting a dataset from two different environments. 
Data from 10 drives will be collected, and each drive will labeled either good or failed, with 4 drives in the good class and 6 drives in the failed class. Drives labeled as good will be from a reliability test, run in a controlled environment by the manufacturer. Drives labeled as failed indicated failure to detect malfunction of device. It should be noted that since the good drive data will be collected in a controlled 
Uniform environment and the failed data come from drives that will be operated by users, it is reasonable 
to expect that there will be differences between the two populations due to the different manner of operation. 
The data will be collected, analyzed and stored using SPSS and Epi-Data statistical applications. 
\subsection{Instrumentation} 
The following will be the instruments we propose to use: 
Surveys. 
Questionnaires. 
Observation grids. 

\subsection{Material or tools} 
The materials used in the survey include questionnaires which will contain structured questions to be answered by targeted users who will input the required information by use of pens, video cameras will be used to capture daily activities and the statistical analysis Applications like EPI-DATA, SPSS will be used to store and analyze data. 


\section{Sample of Questionnaire.}
1.	Do you have a hard disk?
  YES                                    NO
2.	Do you always back up your data?
                   YES                                 NO
3.	If yes how often?
…………………………………………………………………………………………………………..
4.	How do you back up your data?
.........................................................................................
5.	How variable is your data?
………………………………………………………………………………………..
6.	Do you get hard disk failures?
                   YES                                 NO
7.	If yes how do you detect the failure?
…………………………………………………………………………………………………
8.	How often do you replace your hard drives?
     ………………………………………………………………………………………………………
9.	What criteria do you use?
………………………………………………………………………………………………..

	
\section{Procedures} 
We intend to carry out our study using the following sequence of steps: 
1. We will choose a target area for example a school institution in Kampala and request for permission to interact and access information from the respondents for example student or stuff. 
2. Brief the respondents on the aim of the questionnaires or survey to be carried out and how to attempt the questions or participate in the survey. 
3. The respondent will be given a time limit for example one hour to participate in the activity or attempt the questionnaire and finally the data is be collected. 
4. Data entry will then be carried out. 
5. Data will be analyzed using statistical applications like SPSS and manipulated to draw graphs and tables. 

\section{Participants} 
The population from which a sample of subjects is drawn include; people operating in data centers and public institutions like schools etc. in Kampala district for example: 
Computer Technicians since they are the ones who repair and maintain computers.
Database administrators who are responsible for the performance, integrity and security of databases. 
Computer sales personnel who sale computer hardware devices. 
\section{Ethical Consideration} 
The researcher will ensure that all citations and references of different authors are acknowledged. The researcher will maintain confidentiality of respondents and protect their privacy at all times. The researcher will professionally present himself to the respondents as this will affect the attitude and expectations of the respondents. The researcher will use the language that is neutral as possible regarding the terminology when interviewing people and jargons will be avoided. Lastly, respondents will be interviewed on appointment. 
\section{Conclusion:} 
The research study will adopt both qualitative and quantitative approach using both primary and secondary data. Data will mainly be collected through conducting interviews with selected respondents and reviewing previous literature relevant to the topic.
\section{Architecturial Design}
\includegraphics{harddiskfailurepredictor.jpg}
\end{document}

\documentclass[confrence]{IEEEtran}
\begin{document}

\cite{hamerly2001bayesian}


\bibliographystyle{IEEEtran}
\bibliography{refrences}

 

\section{References }

\end{document}


